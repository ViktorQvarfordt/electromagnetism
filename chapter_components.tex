\chapter{Electrical components}


\section{Parallel-plate capacitor}

\emph{This section assumes familiarity with the following concepts.}
\begin{itemize}
\item Capacitance \ref{sec:capacitance}
\item Conductor \ref{def:conductor}
\item Dielectric \ref{def:dielectric}
\end{itemize}

We define the parallel-plate capacitor as two conducting plates
positioned parallel and opposite to each other with equal area and
equal but opposite charge. We say that a capacitor has charge $Q$ when
the plates have charge $Q$ and $-Q$ respectively.

\subsection{Electric field}
\label{capacitor electric field}

If the edge effects of the plates are ignored, the electric field
vanishes outside the capacitor and inside it becomes uniform with
magnitude
\begin{equation*}
  E=\frac{\sigma}{\epsilon_0}
\end{equation*}
\begin{proof}
  By ignoring the edge effects of the plates, the electric field of each
  plate is given by result~\ref{electric field of conducting
    plane}. And blabla it follows.
\end{proof}


\subsection{Potential difference / voltage}
\label{capacitor voltage}

If the plates are separated with distance $d$ the voltage is given by
\begin{equation*}
  V = Ed
\end{equation*}

\begin{proof}
  Let $c$ be a line connecting the two plates, using
  definition~\ref{def:electric potential} and ignoring the sign (see
  the remark), we get
  \begin{equation*}
    V = \int_c \vec{E}\cdot d\vec{l} = E\int_c dl = Ed \qedhere
  \end{equation*}
\end{proof}

\paragraph{Remark} When considering the potential difference between
two objects, like the plates, the sign of the result will differ
depending on which point is taken as the ``starting point''. It is
thus common to let $V$ implicitly refer to $\abs{V}$.


\subsection{Capacitance}

Assume a parallel-plate capacitor with charge $Q$, area $A$ and
separation $d$. The capacitance of the capacitor is given by
\begin{equation*}
  C = \epsilon_0 \frac{A}{d}
\end{equation*}

\begin{proof}
  The results follows directly from the relations
  \begin{align*}
    C \underset{(1)}{=} \frac{Q}{V} && Q \underset{(2)}{=} \sigma A &&
    V \underset{(3)}{=} Ed && E \underset{(4)}{=} \frac{\sigma}{\epsilon_0}
  \end{align*}
  with motivation:
  (1) by definition~\ref{def:capacitance},
  (2) by definition~\ref{charge density},
  (3) by \ref{capacitor voltage},
  (4) by \ref{capacitor electric field}.
\end{proof}


\subsection{Stored energy}

Energy stored in a capacitor is given by
\begin{equation*}
  W_C = \frac{1}{2}QV = \frac{1}{2}CV^2 = \frac{1}{2}\frac{Q}{C}
\end{equation*}

\begin{proof}
  \ldots
\end{proof}


\section{Inductor}

\emph{See section~\ref{sec:inductance} for the concept of
  inductance.}
