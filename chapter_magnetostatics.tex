\chapter{Magnetostatics}

\section{Laws}



\subsection{Lorentz force law}

A charged particle $q$ placed in an electrostatic field $\vec{E}$ and
magnetostatic field $\vec{B}$ will experience a force $\vec{F}$ given
by
\begin{equation*}
  \vec{F} = q\vec{E} + q\vec{v}\times\vec{B}
\end{equation*}

\paragraph{Note} This law is sometimes used as the definition for
$\vec{E}$ and $\vec{B}$, in which case the law is to be understood as
the following empirical statement.
\footnote{\url{http://en.wikipedia.org/wiki/Lorentz_force\#Lorentz_force_law_as_the_definition_of_E_and_B}}
\begin{quote}
  The electromagnetic force $\vec{F}$ on a test charge at a given
  point and time is a certain function of its charge $q$ and velocity
  $\vec{v}$, which can be parameterized by exactly two vectors
  $\vec{E}$ and $\vec{B}$.
\end{quote}

\paragraph{Note} The Lorentz force an be derived in analytical mechanics.\footnote{\url{http://en.wikipedia.org/wiki/Lorentz_force\#Lorentz_force_and_analytical_mechanics}}



\subsection{Ampère's circuital law}

Let $c$ be a closed curve with current $I$ flowing through and
$\vec{j}$ being the current density, then the following holds.
\begin{align*}
  \oint_c \vec{B}\cdot d\vec{l} &= \mu_0I = \mu_0 && \nabla\times\vec{B} = \mu_0\vec{j} \\
  \oint_c \vec{H}\cdot d\vec{l} &= \mu_0I_\text{f} && \nabla\times\vec{H} = \vec{j}_\text{f}
\end{align*}
See <somewhere> for the relation between $\vec{B}$ and $\vec{H}$.



\subsection{Biot--Savart law}

\section{Fundamental concepts and definitions}

\subsection{Magnetization}
