\chapter{Electrical components}


\section{Capacitor -- \normalfont parallel-plate}

\emph{This section assumes familiarity with the following concepts.}
\begin{itemize}
\item Capacitance \ref{sec:capacitance}
\item Conductor \ref{def:conductor}
\item Dielectric \ref{def:dielectric}
\end{itemize}

We define the parallel-plate capacitor as two conducting plates
positioned parallel and opposite to each other with equal area and
equal but opposite charge. We say that a capacitor has charge $Q$ when
the plates have charge $Q$ and $-Q$ respectively.

\subsection{Electric field of parallel-plate capacitor}

If the edge effects of the plates are ignored, the electric field
vanishes outside the capacitor and inside it becomes uniform with
magnitude
\begin{equation*}
  E=\frac{\sigma}{\epsilon_0}
\end{equation*}
\begin{proof}
  By ignoring the edge effects of the plates, the electric field of each
  plate is given by result~\ref{electric field of conducting
    plane}. And blabla it follows.
\end{proof}

\subsection{Capacitance $C$}

Assume a parallel-plate capacitor with charge $Q$, area $A$ and
separation $d$.



\section{Inductor}

\emph{See section~\ref{sec:inductance} for the concept of
  inductance.}
