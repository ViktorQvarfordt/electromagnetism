\chapter{Electrostatics}

Assume that classical mechanics holds. We then define a new property
for all bodies, \emph{charge}, it can take any
\footnote{In reality charges are quantized. However, this is of no
  interest in general, so we proceed with as few assumptions as
  possible. Also, on a macroscopic scale, continuous charge
  distributions are good approximations.}
value in $\mathbb{R}$. It is the effects of this new property that is
studied in this document.

\emph{Note that electrostatics only considers static, non-moving
  charges. See electrodynamics for the non-static case.}




\section{Laws}

Experimental facts, can be considered as the axioms of fundamental
assumptions of the theory.

Note that Coulomb's and Gauss's law should not be considered as
separate laws, see theorem~\ref{thm:coulomb-gauss}.


\subsection{Coulomb's law}

Let $q_1$ and $q_2$ be two point charges, with position $\vec{r}_1$
and $\vec{r}_2$ respectively. Then there is a force exerted between
the particles, given by
\begin{equation*}
  \vec{F}_{12} = \frac{1}{4\pi\epsilon_0}\frac{q_1q_2}{r_{12}^2}\vhat{r}_\text{12}
\end{equation*}


\subsection{Gauss's Law}

Let bla and bla, then
\begin{equation*}
  \oiint_S \vec{E} \cdot d\vec{S} = \frac{Q}{\epsilon_0}
\end{equation*}




\section{Fundamental concepts and definitions}

\subsection{Charge and charge density}

\subsection{Electrostatic field $\vec{E}$}
\label{def:electric field}

Let $Q$ be a point charge and $\vec{r}$ be the position vector
relative to $Q$. We define the electrostatic field $\vec{E}$ caused by
$Q$ as
\begin{equation*}
  \vec{E}(\vec{r}) = \frac{1}{4\pi\epsilon_0}\frac{Q}{r^2}\vhat{r}
\end{equation*}

\paragraph{Motivation} The force $\vec{F}$ exerted by $Q$ on a charge $q$ is
given by $\vec{F} = q\vec{E}$.

\subsection{Electrostatic potential $\phi$}
\label{def:potential}
\begin{equation*}
  \vec{E} = - \nabla\phi
\end{equation*}

\subsection{Electrostatic potential energy $U$}

\subsection{Conductor}
\label{def:conductor}

\subsection{Dielectric}
\label{def:dielectric}

\paragraph{Remark} Insulators are closely related. Dielectrics are
used to \emph{store electrical charges}, while insulators are used to
\emph{block the flow of electric charges}. While all dielectrics are
insulators, all insulators aren't dielectric; because they cannot
store charges, unlike dielectrics.
\footnote{\url{http://wiki.answers.com/Q/What_is_the_Difference_between_dielectric_and_insulator}}




\section{Theorems}


\subsection{Equivalence between Coulomb's and Gauss's laws}
\label{thm:coulomb-gauss}
Gauss's law holds less information than Coulomb's law. Coulomb's law
implies spherical symmetry of $\vec{E}$ around a point charge, Gauss's
law holds no information of $\nabla\times\vec{E}$.

Gauss's law can be derived from Coulomb's law. To derive Coulomb's law
from Gauss's law, one also needs the assumption $\nabla\times\vec{E}=0$.


\subsection{$\vec{E}$ is conservative, has a potential and
  $\nabla\times\vec{E}=0$}
\label{thm:conservative}
The three statements are equivalent. \url{http://en.wikipedia.org/wiki/Conservative_force#Mathematical_description}


\subsection{$\nabla\times\vec{E} = 0$}
\begin{proof}
  By Stokes' theorem we have
  \begin{equation*}
    \oint_c \vec{E}\cdot d\vec{l} = \iint_S (\nabla\times\vec{E})\;d\vec{S}
  \end{equation*}
  By theorem \ref{thm:conservative} ($\vec{E}$ being conservative) we
  have $\oint_c \vec{E}\cdot d\vec{l} = 0$ for any choice of $c$. Choose
  $c$ so that only a point is enclosed by $c$. It follows by Stokes'
  theorem that $\nabla\times\vec{E}=0$ at that point. Since $E$ is
  conservative everywhere, it follows that $\nabla\times\vec{E}=0$
  always holds.
\end{proof}

\begin{proof}[Alternative proof]
  In general we have
  \begin{align*}
    \nabla\times\nabla\phi &= \left(\PDop{x},\PDop{y},\PDop{z}\right)\times\left(\PD{\phi}{x},
    \PD{\phi}{y}, \PD{\phi}{z}\right) \\
    &= \left(
         \PDop{y}\PD{\phi}{z}-\PDop{z}\PD{\phi}{y},
         \PDop{z}\PD{\phi}{x}-\PDop{x}\PD{\phi}{z},
         \PDop{x}\PD{\phi}{y}-\PDop{y}\PD{\phi}{x}
       \right) \\
    &= \vec{0}
  \end{align*}
  Using this and definition \ref{def:potential} ($\vec{E} = - \nabla\phi$),
  we get
  \begin{equation*}
    \nabla\times\vec{E} = \nabla\times(-\nabla\phi) =
    -\nabla\times\nabla\phi = \vec{0}
  \end{equation*}
\end{proof}


\section{General results}

\subsection{Uniform charge distribution}
\label{uniform charge distribution}
The charge distribution of a conductor in the absence of an electric
field is uniform.
\begin{proof}
  Assume a conductor with non-zero net charge. Each charges exerts a
  repelling force on all other charges, forcing the charges to spread
  out uniformly in the conductor.
\end{proof}

\subsection{Electric field of conducting plane}
\label{electric field of conducting plane}

Assume a conducting plane with surface charge density $\sigma$, which
by theorem~\ref{uniform charge distribution} must be uniform. If the
plane is infinitely large, the electric field is constant with magnitude
\begin{equation*}
  E = \frac{\sigma}{2\epsilon_0}
\end{equation*}

\subsubsection{Finite disc}

Assume the conducting plane to be a disc with radius $R$, then the
electric field on distance $z$ above the midpoint of the disk is
\begin{equation*}
  \vec{E}(z) = ...
\end{equation*}
\begin{proof}
  By definition~\ref{def:electric field} we get a contribution
  $d\vec{E}$ to the electric field from a point on conducting plane
  \begin{equation*}
    d\vec{E}(z)=\frac{1}{4\pi\epsilon_0}\frac{\sigma}{r^2}
  \end{equation*}
\end{proof}

\subsubsection{Infinite plane}
\begin{equation*}
  \vec{E}(z) = \frac{\sigma}{2\epsilon_0}\vhat{z}
\end{equation*}
\begin{proof}[Proof by considering an infinite disc]
  Consider the limit $R \to \infty$.
\end{proof}
\begin{proof}[Proof by Gauss's law]
  Cylindrical Gauss surface\ldots
\end{proof}
